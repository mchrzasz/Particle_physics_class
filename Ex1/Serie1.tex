% Please name the set of problems ubgXX.tex, and the solutions lsgXX.tex, where XX is the number

\documentclass[11pt]{article}
\usepackage[german]{babel}
\usepackage{epsfig,graphicx}
\usepackage{amsmath}
%\usepackage{axodraw}
\usepackage{mathrsfs}
\usepackage{amsfonts}
\usepackage{slashed}
%\usepackage{txfonts, pxfonts}
\textheight 240mm
\textwidth 168mm
\topmargin -1.5cm
\oddsidemargin -4mm
\evensidemargin -4mm
\pagestyle{empty}
\parindent 0mm

% Commands for typesetting of the exercises/solutions: please do not edit them
\newenvironment{questions}{\begin{list}{\alph{enumi})}{\usecounter{enumi}\leftmargin6mm}}{\end{list}}
\newcounter{exercise}
\newcommand{\exercise}[1]{\vspace{0.5cm}\stepcounter{exercise} \noindent {\large {\bf \arabic{exercise}. #1}}\\[1mm] }
%\newcommand{\serie}[2]{\noindent{{\bf \mbox{Exercises for Kern- und Teilchenphysik II --- Prof.\,F.\,Canelli, Prof\,N.\,Serra--- \\Spring term 2015 - Exercise sheet 1}}}\\
\newcommand{\serie}[2]{ \bf{Exercises for Kern- und Teilchenphysik II} \\ Prof.\,F.\,Canelli, Prof\,N.\,Serra \\Spring term 2015 - Exercise sheet 1\\
\rule{\linewidth}{0.3mm} \\ \noindent{\it Issued: 27 February 2015\\ Due: 5 March 2015  \\Discussion: 5 March 2015  }\\[2mm]
 }
\newcommand{\solutions}[1]{\noindent{ {\bf \mbox{Einf\"uhrung in die Kern- und Teilchenphysik --- Prof.\,K.\,Kirch --- Serie #1}}}\\\rule{\linewidth}{0.5mm} \noindent{\it L\"osungen }\\[2mm]}


% The following commands are defined for your convenience, they work in both math and text mode
\def\epem{\ensuremath{\mathrm{e^+e^-}}}      % e+e-
\def\TeV{\ensuremath{\mathrm{\ Te\kern -0.1em V}}} % TeV in correct typesetting
\def\GeV{\ensuremath{\mathrm{\ Ge\kern -0.1em V}}} % GeV in correct typesetting
\def\MeV{\ensuremath{\mathrm{\ Me\kern -0.1em V}}} % MeV in correct typesetting
\def\keV{\ensuremath{\mathrm{\ ke\kern -0.1em V}}} % keV in correct typesetting
\def\eV{\ensuremath{\mathrm{\ e\kern -0.1em V}}} % eV in correct typesetting
\def\tev{\TeV{}} % same as above, for convenience
\def\gev{\GeV{}}
\def\mev{\MeV{}}
\def\kev{\keV{}}
\def\ev{\eV{}}
\def\Wp{\ensuremath{\mathrm {W^+}}} % W+ boson
\def\Wm{\ensuremath{\mathrm {W^-}}} % W- boson
\def\ra{\ensuremath{\rightarrow}} %  "GOES TO" arrow for reactions
\def\rts {\ensuremath{\sqrt{s}}} % square root of s
\newcommand{\La}{\ensuremath{\mathcal{L}}}  % Luminosity
\newcommand{\p}{\ensuremath{\partial}} %

\begin{document}
%%%%%%%%%%%%%%%%%%%%%%%%%%%%%%%%%%%%%%%%%%%%%%%%%%%%%%%%%%%%%
%   Uncomment the appropriate line:
%
%\serie{1}{}  % insert (1) problem set number and (2) deadline here
%\serie{1}{1./3./4. M\"{a}rz 2011}  % insert (1) problem set number and (2) deadline here
%
%\solutions{1}  % insert solutions number here
%
%%%%%%%%%%%%%%%%%%%%%%%%%%%%%%%%%%%%%%%%%%%%%%%%%%%%%%%%%%%%%



%\begin{minipage}
%\fontsize{14pt}{12pt}\fontsize{10pt}{12pt}
\large 
\textbf{
\centerline{Kern- und Teilchenphysik II}\\
\centerline{Spring Term 2015}\\
%\centerline{Prof.\,F.\,Canelli  Prof\,N.\,Serra}
\\
\Large \centerline{Exercise Sheet 1}
\\\\
}

\large
Lecturers: Prof.\,F.\,Canelli,  Prof.\,N.\,Serra\\
Assistants: Mr. E.Bowen, Dr. A.de Cosa

%Assistants: B. Casal Lara\~na, Dr. M. Doneg\`a, L. Tancredi, A. Torre\\
%\verb+https://moodle-app2.let.ethz.ch/course/view.php?id=1064+
%\end{minipage}
\normalsize


\exercise{Dirac Matrices}

Show that the Dirac matrices have the following properties:

\begin{questions}

\item { \hfil 
\begin{math}
\{\gamma^{\mu}, \gamma^{\nu}\}= 2g^{\mu\nu}
\end{math} 
}
\item { \hfil 
\begin{math}
\{\gamma^{\mu}, \gamma^{5}\}= 0
\end{math} 
}
\item { \hfil 
\begin{math}
\gamma^{\mu\dagger}\gamma^{0}=\gamma^{0}\gamma^{\mu}
\end{math} 
\begin{flushright}
3 pt
\end{flushright}
}

\end{questions}

\exercise{Dirac Equation}


The Dirac equation for the particle spinor $u(\mathrm{p})$ is:
\begin{displaymath}
(\gamma^{\mu} p_{\mu} - m)~u(\mathrm{p}) = 0
\end{displaymath} 
\begin{questions}
\item Defining $\slashed{p}\equiv \gamma^{\mu} p_{\mu}$, write the equivalent equation for the antiparticle spinor $v(\mathrm{p})$.
\begin{flushright}
1 pt
\end{flushright}
\item Write the Dirac equation for the adjoint of the particle spinor, $\bar{u}(\mathrm{p})$ (which is defined as $\bar{u}(\mathrm{p}) = u^{\dagger}(\mathrm{p})\gamma^{0}$), and for the antiparticle spinor $\bar{v}(\mathrm{p})$.

[$Hint:$ Use the third property of the Dirac matrices listed in the previous exercise]
\begin{flushright}
2 pt
\end{flushright}

\item Verify the orthogonality of particle and antiparticle spinors:
\begin{displaymath}
u^{\dagger(1)}u^{(2)} = 0,~~~~~~v^{\dagger(1)}v^{(2)} = 0
\end{displaymath} 
\begin{flushright}
2 pt
\end{flushright}
\item Show that:
\begin{displaymath}
\bar{u}u = 2mc,~~~~~~\bar{v}v = -2mc
\end{displaymath} 
\begin{flushright}
2 pt
\end{flushright}
\item Show that they are complete:
\begin{displaymath}
\sum_{s=1,2}u^{(s)}\bar{u}^{(s)}= (\gamma^{\mu} p_{\mu} + mc),~~~~~~\sum_{s=1,2}v^{(s)}\bar{v}^{(s)}= (\gamma^{\mu} p_{\mu} - mc)
\end{displaymath} 
\begin{flushright}
4 pt
\end{flushright}
\end{questions}
Considering that $u^{(s)}$ and $v^{(s)}$ are the canonical solutions to the Dirac equation.

\exercise{Conservation of charge}

Show how the conservation of charge can be deduced from the continuity equation:
\begin{displaymath}
\partial_{\mu}J^{\mu} = 0
\end{displaymath} 
\begin{flushright}
2 pt
\end{flushright}

\exercise{Coulomb Scattering}

Consider the case  of Coulomb scattering of a charged spin-0 particle from an external field A, which is a static field of a point charge Ze located in the origin:
\begin{displaymath}
A_{mu} = (V, \bar{A}) = (V,0)
\end{displaymath} 
The potential is given by:
\begin{displaymath}
V(x) = \frac{Ze}{4\pi\mid \mathbf{x}\mid}
\end{displaymath} 
 Compute:
\begin{questions}
\item The transition amplitude $T_{fi}$ for this process:
\begin{displaymath}
T_{fi}=-i\int \mathrm{d}^4x {j^{\mu}_fi}A_{\mu}
\end{displaymath}
$Hint:$ Make use of Fourier Transformation
\begin{displaymath}
\frac{1}{\mid (\bar{p_f} - \bar{p_i}) \mid ^2} = \int \mathrm{d}^3x~\mathrm{e}^{i (\bar{p_f} - \bar{p_i}) \bar{x}} \frac{1}{4\pi\mid \bar{x} \mid}
\end{displaymath}
\begin{flushright}
3 pt
\end{flushright}
\item Its transition probability:
\begin{displaymath}
\omega_{fi}= \lim_{T\to\infty}\frac{\mid T_{fi} \mid ^2}{T}
\end{displaymath}
$Hint:$ Assume that the interaction occurs during a time period T from $t=-T/2$ up to $t=+T/2$
\begin{flushright}
3 pt
\end{flushright}
\item The cross section:
\begin{displaymath}
\mathrm{d}\sigma=\frac{\omega_{fi}}{flux_{i}}\mathrm{d}LIPS
\end{displaymath}
\begin{flushright}
3 pt
\end{flushright}
\item Derive the Rutherford scattering cross section (i.e. consider the non-relativistic limit of the cross section computed in the previous point) 
\begin{flushright}
1 pt
\end{flushright}
\end{questions}

\end{document}
