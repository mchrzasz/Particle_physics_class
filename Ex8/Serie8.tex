% Please name the set of problems ubgXX.tex, and the solutions lsgXX.tex, where XX is the number

\documentclass[11pt]{article}
\usepackage[german]{babel}
\usepackage{epsfig,graphicx}
\usepackage{amsmath}
%\usepackage{axodraw}
\usepackage{mathrsfs}
\usepackage{amsfonts}
\usepackage{xspace, amsmath}
\usepackage{hepparticles}
\usepackage{hepunits}
\usepackage{slashed}
%\usepackage{txfonts, pxfonts}
\textheight 240mm
\textwidth 168mm
\topmargin -1.5cm
\oddsidemargin -4mm
\evensidemargin -4mm
\pagestyle{empty}
\parindent 0mm

% Commands for typesetting of the exercises/solutions: please do not edit them
\newenvironment{questions}{\begin{list}{\alph{enumi})}{\usecounter{enumi}\leftmargin6mm}}{\end{list}}
\newcounter{exercise}
\newcommand{\exercise}[1]{\vspace{0.5cm}\stepcounter{exercise} \noindent {\large {\bf \arabic{exercise}. #1}}\\[1mm] }
%\newcommand{\serie}[2]{\noindent{{\bf \mbox{Exercises for Kern- und Teilchenphysik II --- Prof.\,F.\,Canelli, Prof\,N.\,Serra--- \\Spring term 2015 - Exercise sheet 1}}}\\
\newcommand{\serie}[2]{ \bf{Exercises for Kern- und Teilchenphysik II} \\ Prof.\,F.\,Canelli, Prof\,N.\,Serra \\Spring term 2015 - Exercise sheet 5\\
\rule{\linewidth}{0.3mm} \\ \noindent{\it Issued: 27 February 2015\\ Due: 5 March 2015  \\Discussion: 5 March 2015  }\\[2mm]
 }
\newcommand{\solutions}[1]{\noindent{ {\bf \mbox{Einf\"uhrung in die Kern- und Teilchenphysik --- Prof.\,K.\,Kirch --- Serie #1}}}\\\rule{\linewidth}{0.5mm} \noindent{\it L\"osungen }\\[2mm]}


% The following commands are defined for your convenience, they work in both math and text mode
\def\epem{\ensuremath{\mathrm{e^+e^-}}}      % e+e-
\def\TeV{\ensuremath{\mathrm{\ Te\kern -0.1em V}}} % TeV in correct typesetting
\def\GeV{\ensuremath{\mathrm{\ Ge\kern -0.1em V}}} % GeV in correct typesetting
\def\MeV{\ensuremath{\mathrm{\ Me\kern -0.1em V}}} % MeV in correct typesetting
\def\keV{\ensuremath{\mathrm{\ ke\kern -0.1em V}}} % keV in correct typesetting
\def\eV{\ensuremath{\mathrm{\ e\kern -0.1em V}}} % eV in correct typesetting
\def\tev{\TeV{}} % same as above, for convenience
\def\gev{\GeV{}}
\def\mev{\MeV{}}
\def\kev{\keV{}}
\def\ev{\eV{}}
\def\Wp{\ensuremath{\mathrm {W^+}}} % W+ boson
\def\Wm{\ensuremath{\mathrm {W^-}}} % W- boson
\def\ra{\ensuremath{\rightarrow}} %  "GOES TO" arrow for reactions
\def\rts {\ensuremath{\sqrt{s}}} % square root of s
\newcommand{\La}{\ensuremath{\mathcal{L}}}  % Luminosity
\newcommand{\p}{\ensuremath{\partial}} %

\begin{document}
%%%%%%%%%%%%%%%%%%%%%%%%%%%%%%%%%%%%%%%%%%%%%%%%%%%%%%%%%%%%%
%   Uncomment the appropriate line:
%
%\serie{1}{}  % insert (1) problem set number and (2) deadline here
%\serie{1}{1./3./4. M\"{a}rz 2011}  % insert (1) problem set number and (2) deadline here
%
%\solutions{1}  % insert solutions number here
%
%%%%%%%%%%%%%%%%%%%%%%%%%%%%%%%%%%%%%%%%%%%%%%%%%%%%%%%%%%%%%



%\begin{minipage}
%\fontsize{14pt}{12pt}\fontsize{10pt}{12pt}
\large
\textbf{
\centerline{Kern- und Teilchenphysik II}\\
\centerline{Spring Term 2015}\\
%\centerline{Prof.\,F.\,Canelli  Prof\,N.\,Serra}
\\
\Large \centerline{Exercise Sheet 8}
\\\\
}

\large
Lecturers: Prof.\,F.\,Canelli,  Prof.\,N.\,Serra\\
Assistants: Dr. M.\,Chrzaszcz, Dr. A.de Cosa

%Assistants: B. Casal Lara\~na, Dr. M. Doneg\`a, L. Tancredi, A. Torre\\
%\verb+https://moodle-app2.let.ethz.ch/course/view.php?id=1064+
%\end{minipage}
\normalsize

Issued: 11 May 2015 \\
Due: 22 May 2015  \\ 
Discussion: 22 May 2015 \\

\exercise{Helicity suppression}

Calculate the branching fractions of the decay of $K^- \to \mu^- \bar{\nu}_{\mu}$ and $K^- \to e^- \bar{\nu}_{e}$. Compare the ratio to the experimental measured one.
\begin{flushright}
1 pt
\end{flushright}

\exercise{CKM matrix}

Show that if the CKM matrix is unitary the GIM mechanism for eliminating the $K^0 \to \mu \mu$ works for any number of generations.
\begin{flushright}
3 pt
\end{flushright}
Draw a Feyman diagram that allows the production of $K \to \mu \mu$, $B_s \to \mu \mu$.
\begin{flushright}
1 pt
\end{flushright}
How many independent parameters are there in a general $3 \times 3$ unitary matrix.
\begin{flushright}
2 pt
\end{flushright}
Generalize the solution to $n \times n$ unitary matrix.
\begin{flushright}
1 pt
\end{flushright}
Can you reduce the CKM matrix to only real elements? What if we would have just 2 generations of quarks?
\begin{flushright}
3 pt
\end{flushright}

Lets parametrize the CKM matrix using $\theta_{12}$, $\theta_{23}$, $\theta_{13}$, $\delta$. Show that for any real values of this parameters the matrix is unitary.
\begin{flushright}
3 pt
\end{flushright}


\exercise{$Z$ boson}
Predict in the mass of $Z$ and $W$ bosons in Glashow-Weinberg-Salam model. Experimental values for $G_F$ and $\theta_W$ take from PDG 2014. Was it possible to predict this masses before the LEP?
\begin{flushright}
2 pt
\end{flushright}
Calculate the decay rates for $Z^0 \to f \bar{f}$, where $f$ is any fermion. Assume $m_f<<m_Z$. [Hint: use completeness relation.]
\begin{flushright}
5 pt
\end{flushright}

Give the branching fraction (numerical values) for each of the decay modes valid for your above formula.
\begin{flushright}
2 pt
\end{flushright}
Calculate the ration of: $\dfrac{e^+ e^- \to Z \to \rm{hadrons}}{e^+ e^- \to Z \to \mu \mu}$.
\begin{flushright}
1 pt
\end{flushright}



\exercise{$\tau$ life time 2}
In sheet 5 we estimated $\tau$ lepton life time using leptonic decays of $\tau$. Please repeat the exercise including: $\tau \to d \bar{u} \nu_{\tau}$ and    $\tau \to s \bar{u} \nu_{\tau}$. Compare to experimental values.
\begin{flushright}
3 pt
\end{flushright}


\iffalse
\exercise{Electron neutrino scattering}

Calculate the differential and total cross section for $e^- bar{nu}_{\mu} \to e^- bar{nu}_{\mu}$ in  Glashow-Weinberg-Salam model.
\begin{flushright}
5 pt
\end{flushright}
Is the calculation the same for: $e^- bar{nu}_{e} \to e^- bar{nu}_{e}$?
\begin{flushright}
1 pt
\end{flushright}
\fi

\end{document}
